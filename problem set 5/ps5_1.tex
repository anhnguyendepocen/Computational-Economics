\documentclass[11pt]{article}

\usepackage[lf]{Baskervaldx} % lining figures
\usepackage[bigdelims,vvarbb]{newtxmath} % math italic letters from Nimbus Roman
\usepackage[cal=boondoxo]{mathalfa} % mathcal from STIX, unslanted a bit
\renewcommand*\oldstylenums[1]{\textosf{#1}}

\usepackage[framed]{matlab-prettifier}

\usepackage{amsmath}
\usepackage{amssymb}
\usepackage{parskip}
\usepackage{bm}
\usepackage[a4paper,hmargin=1in,vmargin=1in]{geometry}

\usepackage{graphicx}
\usepackage{color}

\usepackage[hyphens]{url}
\usepackage[colorlinks,linkcolor=red,anchorcolor=black,citecolor=blue]{hyperref}

\newcommand{\alp}{\ensuremath{\alpha}}
\newcommand{\bet}{\ensuremath{\beta}}
\newcommand{\tht}{\ensuremath{\theta}}

\begin{document}

\section*{Problem 1}

The distribution of $x$ is unspecified in the question, and here we assume that $x \sim N(0,1)$.

With Gaussian-Hermite integration, when \# nodes = $[2, 3, 4, 5, 7]$, the numerical results are:
\begin{align*}
& E[X^4] = [1.000, 3.000, 3.000, 3.000, 3.000]\\
& E[X^6] = [1.000, 9.000, 15.000, 15.000, 15.000]\\
& E[1/(1+X^2)] = [0.500, 0.750, 0.600, 0.692, 0.672]
\end{align*}
With Monte Carlo integration, when $n = [10^2, 10^3, 10^4, 5*10^5]$, the numerical results are:
\begin{align*}
& E[X^4] = [2.802, 2.963, 2.946, 3.012]\\
& E[X^6] = [11.098,  15.096, 14.270, 14.705]\\
& E[1/(1+X^2)] = [0.649, 0.669, 0.653, 0.655]
\end{align*}

\appendix

\section*{Matlab codes}

\subsection*{Problem 1}

\begin{lstlisting}[style=Matlab-editor]
clc; clear all; close all;

%% Initialization
syms x;
y = [x^4, x^6, 1/(1+x^2)]';
my = matlabFunction(y);

%% Guassian-Hermite Quadrature Integration
n= [2, 3, 4, 5, 7];
for i = 1:length(n)
    [x, w] = qnwnorm(n(i), 0, 1); % by Miranda and Fackler (2002)
    int = sum(repmat(w',3,1).*my(x'),2);
    disp(sprintf('When n = %d, GH integrations are %4.3f, %4.3f, %4.3f', n(i), int));
end

%% Monte Carlo Integration
n = [1e2, 1e3, 1e4, 5*1e4];
for i =1:length(n)
    r = randn(1,n(i));
    int = sum(my(r),2)./n(i);
    disp(sprintf('When n = %d, MC integrations are %4.3f, %4.3f, %4.3f', n(i), int));
end
\end{lstlisting}

\end{document}

